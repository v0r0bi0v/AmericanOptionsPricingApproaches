\documentclass[8pt]{article}

\usepackage[russian,english]{babel}

\usepackage{cmap}
\usepackage[T2A]{fontenc}
\usepackage[utf8]{inputenc}
\usepackage{amsfonts}
\usepackage{amsthm}
\usepackage{mathtools}
\usepackage{color}
\usepackage{hyperref}
\usepackage{graphicx}
\usepackage{pdfpages}
\usepackage{forest}
\usepackage{adjustbox}
\usepackage{times}
\usepackage{tikz}

\usepackage{minted} % для кода
\usepackage{listings} % для кода

\usepackage[style=numeric,backend=biber,maxbibnames=99]{biblatex}
\addbibresource{paper.bib}

\renewcommand*{\finalnamedelim}{\addcomma\space}
\renewcommand*{\finalnamedelim}{\addcomma\space}

\lstset{ % для кода
    language=Python,
    basicstyle=\ttfamily\small,
    keywordstyle=\color{blue},
    commentstyle=\color{green},
    stringstyle=\color{red},
    numbers=left,
    numberstyle=\tiny\color{gray},
    stepnumber=1,
    numbersep=5pt,
    backgroundcolor=\color{white},
    showspaces=false,
    showstringspaces=false,
    showtabs=false,
    tabsize=4,
    captionpos=b,
    breaklines=true,
    breakatwhitespace=true,
    escapeinside={\%*}{*)}
}

% \mode<presentation>{
%     \usetheme{Marburg}
%     \usecolortheme{sidebartab}
% }

% \newtheorem{theorem}{Theorem}[section]
% \newtheorem{lemma}{Lemma}[section]
% \newtheorem{proposition}{Proposition}[section]
% \newtheorem{corollary}{Corollary}[section]
% \newtheorem{definition}{Definition}[section]
% \newtheorem{remark}{Remark}[section]
% \newtheorem{example}{Example}[section]

\newcommand{\E}{\ensuremath{\mathbb{E}}}
\newcommand{\D}{\ensuremath{\mathbb{D}}}
\renewcommand{\C}{\ensuremath{\mathbb{C}}}
\newcommand{\R}{\ensuremath{\mathbb{R}}}
\newcommand{\Q}{\ensuremath{\mathbb{Q}}}
\newcommand{\Z}{\ensuremath{\mathbb{Z}}}
\renewcommand{\|}{\ensuremath{\hspace{0.1cm} | \hspace{0.1cm}}}

\newcommand{\red}[1]{\textcolor{red}{#1}}
\newcommand{\blue}[1]{\textcolor{blue}{#1}}
\newcommand{\green}[1]{\textcolor{green}{#1}}
\newcommand{\orange}[1]{\textcolor{orange}{#1}}
\newcommand{\teal}[1]{\textcolor{teal}{#1}}
\newcommand{\purple}[1]{\textcolor{purple}{#1}}

\newcommand{\todo}{\red{TODO}}

\renewcommand{\phi}{\varphi}
\renewcommand{\epsilon}{\varepsilon}
\renewcommand{\le}{\leqslant}
\renewcommand{\ge}{\geqslant}


\begin{document}
    \title{American Options Pricing Approaches}

    \date{\today}

    \maketitle

    \begin{abstract}
    \todo
\end{abstract}

    \tableofcontents

    \section{Introduction}
        \todo

    \section{Least-Squares Monte Carlo \cite{LSMC}}
        \todo

    \section{RL methods}
        \subsection{Least-Squares Policy Iteration \cite{LSPI}}
    \todo
\subsection{Other RL methods}  % тут мб новый какой-то придумаем
    \todo

    \section{Upper Bound (via dual problem)}
        \todo


    \section{RL vs LSMC}
        \subsection{Mathematical intuition}
    \todo
\subsection{Convergrence testing}
    \todo

    \appendix
    \section{Special case: Loan Pricing (with early prepayment option)}
        \todo

    \section{Improving convergence}
        \subsubsection{Moment Matching}
    \todo
\subsubsection{Negative Sampling}
    \todo
\subsubsection{Quasi-Random (Sobol sequences)}
    \todo


    \section{Code structure}
        \todo


    \nocite{*}
    \printbibliography
\end{document}
